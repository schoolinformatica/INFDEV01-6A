\section*{Module description}
\begin{tabularx}{\textwidth}{|>{\columncolor{lichtGrijs}} p{.26\textwidth}|X|}
	\hline
	\textbf{Module name:} & \modulenaam\\
	\hline
	\textbf{Module code: }& \modulecode\\
	\hline
	\textbf{Study points \newline and hours of effort for full-time students:} & This module gives \stdPunten, in correspondance with 112 hours:
	\begin{itemize}
		\item 2 x 8 hours frontal lecture
		\item 3 x 8 hours self-study for the theory
		\item the rest is self-study for the practicum
	\end{itemize} \\
	\hline
	\textbf{Examination:} & Written examination and practical assignment (with oral check) \\
	\hline
	\textbf{Course structure:} & Lectures \\
	\hline
	\textbf{Prerequisite knowledge:} & Object oriented programming \\
	\hline
	\textbf{Learning tools:} & \begin{itemize}
			\item Book: \textit{Algorithms} (4rd edition); authors R. Sedgewick, K. Wayne
			\item Lesson slides (pdf): found on N@tschool
			\item Assignments, to be done at home (pdf): found on N@tschool
		\end{itemize} \\
	\hline
	\textbf{Connected to \newline competences:} & \begin{itemize}
			\item Realisation
			\item Analysis
		\end{itemize} \\
	\hline
	\textbf{Learning objectives:} &
		At the end of the course, the student can:
			\begin{itemize}
				\item do performance analysis [\texttt{PERF}]
				\item implement and analyse 
					\begin{itemize}
						\item basic data structures [\texttt{DS}\textsuperscript{I}, \texttt{DS}\textsuperscript{A}]
						\item sorting algorithms [\texttt{SORT}\textsuperscript{I}, \texttt{SORT}\textsuperscript{A}]
						\item recursive data structures [\texttt{REC}\textsuperscript{I}, \texttt{REC}\textsuperscript{A}]
						\item algorithms on graphs [\texttt{GRAPH}\textsuperscript{I}, \texttt{GRAPH}\textsuperscript{A}]
					\end{itemize}
			\end{itemize} \\
	\hline

%\end{tabularx}
%\begin{tabularx}{\textwidth}{|>{\columncolor{lichtGrijs}} p{.26\textwidth}|X|}
%	\hline
	\textbf{Course owners:} & \author\\
	\hline
	\textbf{Date:} & \today \\
	\hline
\end{tabularx}
\newpage
